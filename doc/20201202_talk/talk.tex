%\nonstopmode
\documentclass[aspectratio=169]{beamer}
\usepackage[utf8]{inputenc}
% \usepackage[frenchb]{babel}
\usepackage{amsmath}
\usepackage{mathtools}
\usepackage{breqn}
\usepackage{multirow}
\usetheme{Luebeck}
\usepackage{graphicx}
%\useoutertheme[footline=authortitle,subsection=false]{miniframes}

\beamertemplatenavigationsymbolsempty
\setbeamertemplate{footline}
{%
  \leavevmode%
  \hbox{\begin{beamercolorbox}[wd=.15\paperwidth,ht=2.5ex,dp=1.125ex,leftskip=.3cm,rightskip=.3cm plus1fill]{author in head/foot}%
    \usebeamerfont{author in head/foot} \insertframenumber{} / \inserttotalframenumber
  \end{beamercolorbox}%
  \begin{beamercolorbox}[wd=.2\paperwidth,ht=2.5ex,dp=1.125ex,leftskip=.3cm plus1fill,rightskip=.3cm]{author in head/foot}%
    \usebeamerfont{author in head/foot}\insertshortauthor
  \end{beamercolorbox}%
  \begin{beamercolorbox}[wd=.65\paperwidth,ht=2.5ex,dp=1.125ex,leftskip=.3cm,rightskip=.3cm plus1fil]{title in head/foot}%
    \usebeamerfont{title in head/foot}\insertshorttitle~--~\insertshortdate
  \end{beamercolorbox}}%
  \vskip0pt%
}

\usepackage{tabu}
\usepackage{multicol}
\usepackage{vwcol}
\usepackage{stmaryrd}
\usepackage{graphicx}

\usepackage[normalem]{ulem}

\title[Garage : jouer dans la cour des grands quand on est un hébergeur associatif]{Garage : jouer dans la cour des grands \\quand on est un hébergeur associatif}
\subtitle{(ou pourquoi on a décidé de réinventer la roue)}
\author[Q. Dufour \& A. Auvolat]{Quentin Dufour \& Alex Auvolat}
\date[02/12/2020]{Mercredi 2 décembre 2020}

\begin{document}

\begin{frame}
\titlepage
\end{frame}

\begin{frame}
	\frametitle{La question qui tue}
	
	\begin{center}
		\includegraphics[scale=3]{img/sync.png} \\
		\Huge Pourquoi vous n'hébergez pas vos fichiers chez vous ? \\
	\end{center}

\end{frame}

\begin{frame}[t]
	\frametitle{La cour des grands}

\begin{columns}[t]
\begin{column}{0.5\textwidth}
	{\huge Le modèle du cloud...}
 
\begin{center}
	\includegraphics[scale=0.08]{img/cloud.png}
\end{center}

	+ \underline{intégrité} : plus de perte de données

	+ \underline{disponibilité} : tout le temps accessible
	
	+ \underline{service} : rien à gérer

	\vspace{0.15cm}
	\textbf{changement des comportements}
\end{column}
\pause
\begin{column}{0.5\textwidth}
	{\huge ...et son prix}
 
\begin{center}
	\includegraphics[scale=0.07]{img/dc.jpg}
\end{center}

	- matériel couteux et polluant

	- logiciels secrets

	- gestion opaque

	\vspace{0.2cm}
	\textbf{prisonnier de l'écosystème}
\end{column}
\end{columns}
\end{frame}

\begin{frame}[t]
	\frametitle{Garage l'imposteur}

\begin{columns}[t]
\begin{column}{0.5\textwidth}
	{\huge Ressemble à du cloud...}
 
\begin{center}
	\includegraphics[scale=0.5]{img/shh.jpg}
\end{center}

+ \underline{compatible} avec les apps existantes

+ \underline{fonctionne} avec le mobile

+ \underline{s'adapte} aux habitudes prises


\end{column}

\pause
\begin{column}{0.5\textwidth}
	{\huge ...fait du P2P}

\begin{center}
	\includegraphics[scale=1]{img/death.jpg}
\end{center}

\vspace{0.4cm}

+ \underline{contrôle} de l'infrastructure

+ \underline{transparent} code libre 

+ \underline{sobre} fonctionne avec de vieilles machines à la maison
\end{column}
\end{columns}

\end{frame}
\end{document}

%% vim: set ts=4 sw=4 tw=0 noet spelllang=fr :
